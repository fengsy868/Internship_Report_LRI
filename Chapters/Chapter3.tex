% Chapter 3

\chapter{State of the Art} % Main chapter title here

\label{Chapter3} % For referencing the chapter elsewhere, use \ref{Chapter3}

\lhead{Chapter 3. \emph{State of the Art}} % This is for the header on each page perhaps a shortened title
%----------------------------------------------------------------------------------------
Recently, some scientist use deep learning network on the EEG data to do the emotion recognition, also some other domain adaptation methods have been proposed over the recent years, I will list the abstract of this papers in this chapter to give you the state of the art for this subject.

\section{EEG-based emotion classification using deep belief networks}
\textbf{Abstract:}\cite{zheng2014eeg}
In recent years, there are many great successes in using deep architectures for unsupervised feature learning from data, especially for images and speech. In this paper, we introduce recent advanced deep learning models to classify two emotional categories (positive and negative) from EEG data. We train a deep belief network (DBN) with differential entropy features extracted from multichannel EEG as input. A hidden markov model (HMM) is integrated to accurately capture a more reliable emotional stage switching. We also compare the performance of the deep models to KNN, SVM and Graph regularized Extreme Learning Machine (GELM). The average accuracies of DBN-HMM, DBN, GELM, SVM, and KNN in our experiments are 87.62\%, 86.91\%, 85.67\%, 84.08\%, and 69.66\%, respectively. Our experimental results show that the DBN and DBN-HMM models improve the accuracy of EEG-based emotion classification in comparison with the state-of-the-art methods.


\section{EEG-Based Emotion Recognition Using Deep Learning Network with Principal Component Based Covariate Shift Adaptation}
\textbf{Abstract:}\cite{jirayucharoensak2014eeg}
Automatic emotion recognition is one of the most challenging tasks. To detect emotion from nonstationary EEG signals, a
sophisticated learning algorithm that can represent high-level abstraction is required. This study proposes the utilization of a deep
learning network (DLN) to discover unknown feature correlation between input signals that is crucial for the learning task.The DLN
is implemented with a stacked autoencoder (SAE) using hierarchical feature learning approach. Input features of the network are
power spectral densities of 32-channel EEG signals from 32 subjects. To alleviate overfitting problem, principal component analysis
(PCA) is applied to extract the most important components of initial input features. Furthermore, covariate shift adaptation of
the principal components is implemented to minimize the nonstationary effect of EEG signals. Experimental results show that the
DLN is capable of classifying three different levels of valence and arousal with accuracy of 49.52% and 46.03%, respectively. Principal
component based covariate shift adaptation enhances the respective classification accuracy by 5.55% and 6.53%. Moreover, DLN
provides better performance compared to SVM and naive Bayes classifiers.

\section{Domain Adaptation via Transfer Component Analysis}
\textbf{Abstract:}\cite{pan2011domain}
Automatic emotion recognition is one of the most challenging tasks. To detect emotion from nonstationary EEG signals, a
sophisticated learning algorithm that can represent high-level abstraction is required. This study proposes the utilization of a deep
learning network (DLN) to discover unknown feature correlation between input signals that is crucial for the learning task.The DLN
is implemented with a stacked autoencoder (SAE) using hierarchical feature learning approach. Input features of the network are
power spectral densities of 32-channel EEG signals from 32 subjects. To alleviate overfitting problem, principal component analysis
(PCA) is applied to extract the most important components of initial input features. Furthermore, covariate shift adaptation of
the principal components is implemented to minimize the nonstationary effect of EEG signals. Experimental results show that the
DLN is capable of classifying three different levels of valence and arousal with accuracy of 49.52% and 46.03%, respectively. Principal
component based covariate shift adaptation enhances the respective classification accuracy by 5.55% and 6.53%. Moreover, DLN
provides better performance compared to SVM and naive Bayes classifiers.

\section{Domain Adaptation for Large-Scale Sentiment Classification: A Deep Learning Approach}
\textbf{Abstract:}\cite{glorot2011domain}
The exponential increase in the availability of online reviews and recommendations makes sentiment classification an interesting topic in academic and industrial research. Reviews can span so many different domains that it is difficult to gather annotated training data for all of them. Hence, this paper studies the problem of domain adaptation for sentiment classifiers, hereby a system is trained on labeled reviews from one source domain but is meant to be deployed on another. We propose a deep learning approach which learns to extract a meaningful representation for each review in an unsupervised fashion. Sentiment classifiers trained with this high-level feature representation clearly outperform state-of-the-art methods on a benchmark composed of reviews of 4 types of Amazon products. Furthermore, this method scales well and allowed us to successfully perform domain adaptation on a larger industrial-strength dataset of 22 domains.