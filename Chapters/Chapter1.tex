% Chapter 1

\chapter{Introduction} % Main chapter title

\label{Chapter1} % For referencing the chapter elsewhere, use \ref{Chapter1}

\lhead{Chapter 1. \emph{Introduction}} % This is for the header on each page - perhaps a shortened title

%----------------------------------------------------------------------------------------

Deep learning is part of a broader family of machine learning methods based on learning representations of data. In the recent years, Artificial Neural Nerwork has achieved a success in the domains like Computer Vision, Natural Language Processing. 

Research in this area attempts to make better representations and create models to learn these representations from large-scale unlabeled data. Some of the representations are inspired by advances in neuroscience and are loosely based on interpretation of information processing and communication patterns in a nervous system, such as neural coding which attempts to define a relationship between various stimuli and associated neuronal responses in the brain.\cite{olshausen1996emergence}

In the field of neuroscience, the data of electroencephlogram data(EEG Data) are being used in the brain computer interface(BCI), for example, using the EEG data to control the wheelchair or the gamestick. But they are facing the problem that the EEG data has a high variability problem. The data is not only depends on the subject, but also depends on the session. Which is to say that the EEG data collected in the morning and in the evening from the same subject is different. So it asks to train a model depends on a subject and then to adjust the model for different session.

Domain Adaptation\cite{bridle1990recnorm}\cite{ben2010theory} is a field associated with machine learning and transfer learning. This scenario arises when we aim at learning from a source data distribution a well performing model on a different (but related) target data distribution. For instance, one of the tasks of the common spam filtering problem consists in adapting a model from one user (the source distribution) to a new one who receives significantly different emails (the target distribution). Note that, when more than one source distribution is available the problem is referred to as multi-source domain adaptation.\cite{crammer2008learning}

In this internship, I work under the supervision of Dr. Michele Sebag and Dr. Gaetan Marceau Caron at Laboratoire de Recherche en Informatique (LRI), Gif-sur-Yvette, France. We aim at using domain adaptation to eliminate the variability of the EEG data. 


%-----------------------------------------------------------------------------------------
\section{Outline}
In the first part( \autoref{Chapter2}) the contexte and the objectif of this internship will be introduced, the detail of EEG data, domain adaptation will be explained.

\autoref{Chapter3} will show the state of the art on the field of deep learning or more presicely the domain adaptation.

The following chapter( \autoref{Chapter4}) focuses on the framework of this internship, the program I used to train the model.

\autoref{Chapter5} will give you the experiments I have done and the results achieved. Some comments will also follow the results.

The final \autoref{Chapter6} will be the discussion of the experiments results of using domain adaptation on the EEG data to eliminate the variability problem. Then the perspective of the subject will be given.